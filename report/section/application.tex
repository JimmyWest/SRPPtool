\section{Secure Remote Pair Programming (SRPP) applicaiton}

\subsection{General Information}

The application is a Remote Pair Programming tool which uses encryption for messages that are sent over Internet. This application have been developed and implemented by us. It has the feature to edit source files in a collaborative manner. The core structure are based on the client-server model. A server can be started on the machine where the source code are located. Clients can then connect to the server and open and edit files. Several clients can open and edit the same file in real time. 

The server are implemented in Erlang, since Erlang are suited for server applications. the server loads files in a specific folder into memory where changes are made. Files can then be saved on a client save request, then are the files copied from memory down to disk again. the server handles all connection to clients, and acts as a multicaster for file changes made by clients. If a client requests a file change, mostly in source code, the server will multicast this changes to all connected clients. Clients are grouped by the files they have opened. This way can the same server be used by many clients, which means that many Pairs of programmers, which are working in the same project on the same source files can work on the same server. 

The client are implemented in java, since java are suited for Graphical User Interfaces (GUI) and Java are platform independent. The client can connect to a server, list files on the server and open files, which then can be edited or just viewed. 

File changes the clients can make are based on lines. Each client can changes the content of a specific line which then are distributed by the server to all clients. The choice to use changes of line as the resolution is based on how source code versioning often works. For example in Git and SubVersion are lines in a source code replaced by a new line. This unfortunately introduce a problem with change conflicts, when two clients are changing the same line. But a solution to this problem are not covered in this project.  


\subsubsection{Platform settings}

Server - Erlang R16 or later (including the crypto module). 
Client - Java 1.7 or later (with the javax.crypto package). 


\subsection{Communication}

The communication is between the server and the clients. Every client has a main communication to the server and one communication link for each file. 

An example scenario: There are three clients connected to the server, client one has two files open and client two has four files open and client three has one file open. For this case,client one has three communication links, client two has five communication links and client three has two communication links open. In this scenario, the server holds a total of 10 communication links.
